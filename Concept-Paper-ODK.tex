\documentclass[options]{article}

 \usepackage[
    top    = 2.75cm,
    bottom = 2.50cm,
    left   = 4.00cm,
    right  = 3.50cm]{geometry}

\usepackage[parfill]{parskip}

\title{Development of an advertising form to improve on the beaucracy and delay in advert booking in media houses.}
\author{Walugembe Martin Alvin (216005016, 16/U/12262/EVE) \thanks{supervisor: Ernest Mwebaze}}
\date{%
    Makerere University\\%
    Feb 23, 2018
}


\begin{document}
\begin{titlepage}
\maketitle
\end{titlepage}





\section{\textbf{ Introduction}} 
Most media houses in Uganda are having various products they sell to the public for example newspaper adverts, television adverts as well as radio adverts which has improved their marketability and rate of return on investment.
The basic problem that is currently being faced is the issue of automation of the booking process of these adverts to see to it that there is certainity and accuracy in the booking process and assurance to the clients that their adverts are booked in time. In this regard, the vast use of paper work needs to be checked.\bigbreak
 
\subsection{\textbf{Research Background}}
Due to the vast use of paperwork, various media houses have engaged their procurement and ICT Departments to see to it that the problem is checked.
Through that engagement, ustomer care advisors and assistants have been hired to help out in the follow up of these adverts.
 

 \bigbreak



\subsection{\textbf{Problem Statement}}
Development of an advertising form in various media houses is very important to the clients and staff that are participants in the booking process to avoid the delays, wrong bookings and misappropriate timing in the advert booking process.

\subsection{\textbf{Objectives}}


\subsubsection{\textbf{Main Objective}} 
 To come up with an advertising form that can capture the different types of adverts alongside their dates of run and time in their different categories.

\subsubsection{\textbf{Specific Objectives}}

\begin{itemize}
\item To ensure accuracy in the category of advert being booked
\item To see to it that the adverts are booked in time
  \item To ensure that the adverts booked have run.
  \item To ensure that the form can be accessed by everyone simply
\end{itemize}


\subsection{\textbf{Scope}}
This research is aimed at high school students of boarding schools mainly in wakiso district.

\subsection{\textbf{Purpose of the Study}}
The purpose of this study is to determine the effects of drug abuse in high school students on
Academic performance among high school sophomores and juniors. The discussion of the literature revealed that there is no clear consensus regarding the effect that athletic participation has on academic performance. Further, previous research has identified other problems that have been traced back to athletic performance.


\section{\textbf{Methodology}}
This cross-sectional study was conducted out in 20  schools in Wakiso district. A researcher-made questionnaire was developed to determine knowledge, attitude, and practice of high school students regarding addictive drugs and their associated causes. This was accomplished by recruiting 300 students who were selected by multi-stage random cluster sampling.

\section{\textbf{Results}}
The designed questionnaire identified the status quo of drug abuse according to age, gender, and different schools in wakiso district. We also accessed information about the type of abused drug, the most common causes of drug abuse for the first time, the most important causes of drug abuse, mean age of abusers and mean age at the first abuse, common time and locations of drug abuse, and the most common routes of drug abuse according to gender.

\section{\textbf{Conclusion}}
According to the obtained results, the designed questionnaire is capable to assess the drug abuse status among high school students of boarding schools in wakiso district. Regarding the importance of teenage years in forming the future behaviors of adolescents and the opportunities provided at schools, it is prudent to pay more attention to interventions in this age group in order to increase their knowledge and correct their attitude toward illegal drugs and strengthening their confidence in this regard. These interventions can have an important role in decreasing the rate of drug abuse in this age group
and consequently in the whole community.

\begin{thebibliography}{10} \bibitem{latexGuide}Ministry Of Education article , \emph{M.O.E(June 2015)} \end{thebibliography}



\end{document}