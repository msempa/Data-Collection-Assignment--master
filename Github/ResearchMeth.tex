\documentclass[11pt,A4paper]{article}
\usepackage{zed-csp,graphicx,color}%from
\begin{document}
\begin{titlepage}
  \begin{figure}[h]
  \centerline{\small MAKERERE 
  \includegraphics[width=0.1\textwidth]{muk_log} UNIVERSITY}
\end{figure}
\centerline{COLLEGE OF COMPUTING AND INFORMATIC SCIENCES}
\paragraph{•}
\centerline{DEPARTMENT OF COMPUTER SCIENCE\\}
\paragraph{•}

\centerline{COURSEWORK: RESEARCH METHODOLOGY(BIT 2207)\\}
\paragraph{•}

\centerline{LECTURER: MR.ERNEST MWEBAZE}

\paragraph{•}
\centerline{COMPILED BY: \
 KAKOOZA ALLAN KLAUS}
\paragraph{•}
\centerline{STUDENT NUMBER : 216007552}
\paragraph{•}
\centerline{REGISTRATION NUMBER:16/U/5230/EVE}
\paragraph{•}
\end{titlepage}
\pagenumbering{roman}
\tableofcontents
\newpage
\pagenumbering{arabic}
\section{TOPIC}
Improvement of Traffic Congestion
in Kampala City using a Traffic Sustainability System .
\section{INTRODUCTION}
In Uganda, the road transport being a major player
in promoting economic and social development
compared to other modes of transport like
air, water and railway. Improving the traffic congestion
in the city is one way of increasing productivity
considering the time people spend in
traffic jams. Expanding the transport sector is an
important factor in achieving poverty eradication,
sustainable economic growth and improvement
of public service delivery.
\section{PROBLEM STATEMENT}
 Over the years, highways have been used as major transportation links between cities and
different parts of the city, including the city of Kampala. However, in the recent past, traffic
congestion in Kampala has overwhelmed the highways and Jinja highway has not been spared.
The dual role of the highway as a national and international road has resulted to conflict between
through traffic and city centre traffic. This problem becomes worse during peak hours of each
working day curbed with presence of pedestrian traffic that conflicts with vehicular traffic.
The implementation of the proposed bypasses and ring roads has taken too long. The main
objective of these bypasses i.e. to ease traffic congestionby diverting through traffic
from Jinja Highway and Entebbe road, may not be easily realized since the suburbs, through
which they were designed to pass, are already built up and their own traffic is building up. 
\section{MAIN OBJECTIVE}
To examine the possible causes of poor road network, effects of traffic congestion and possible Solution to solve the problem of congestion in Kampala City. 

\section{SPECIFIC OBJECTIVES}
To improve discipline and Law Implementation because drivers and other road users often are not trained sufficiently to follow lane discipline.\\ \\
To improve roads that cause congestion that results into accidents.It’s not all accidents of automobile are resulted from driver’s error, for stance over speeding, texting while driving, drink and drive, inattentiveness, etc.  but sometimes roads itself are to blame.  

\section{METHODOLOGY}
\subsection{Research Design}
A research design refers to systematic plan drawn by the researcher during the research study. Generally the kind of data used in this study was both quantitative and analytically.
\subsection{Population Size}
According to Uganda Bureau of Statistics, the urban population in Uganda has almost doubled from 2002 which was reported to be over 2.9 million people and now is reported to be around 4.8 million.
\subsection{Sampling Frame}
The sampling frame was  Kampala City and some neighbouring busy roads for example Jinja Rd and Masaka Rd. 
\subsection{Research Procedure}
\subsection{Desk Study}
The study mainly considered reports from Ministry of Transport and Works.
\subsection{Data collection methods}
Various types of data/variables will be needed in order to successfully carry out the study. There
will be both units of observation and units of analysis.
\subsection{Data Processing and Analysis}

All collected information from the survey was recorded, checked and verified for the analysis. Results were then presented using pie charts, graphs and tables to interpret variations and relationships between the genders

\section{REFERENCES}
The following documents informed the development
of this paper:\\ \\
Adam Smith Int. (2005) A Study of urban Transport
Institutional, Financial and Regulatory Frameworks
in large Sub Saharan African Cities; Executive
Summary, November.\\ \\
Godfrey.o.Wandera (2014) Urban traffic congestion
in African cities – an overview of Kampala,
Ministry of Works and Transport, Uganda

\end{document}